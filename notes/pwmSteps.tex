\documentclass[12pt]{article}
\usepackage{enumitem}
\usepackage{geometry}
\usepackage{parskip}
\geometry{margin=1in}

\title{PWM Setup on STM32F411}
\author{}
\date{}

\begin{document}

\maketitle

\section*{Steps to Use a PWM Pin on STM32F411}

\begin{enumerate}[label=\textbf{Step \arabic*:}]
    \item \textbf{Select a Timer and Channel} \\
    Choose a timer (e.g., TIM1, TIM2, etc.) that supports PWM output. Then select a channel on that timer (typically Channel 1 to 4).

    \item \textbf{Identify a PWM-capable GPIO Pin} \\
    Use the STM32 alternate function mapping table to find a GPIO pin that supports the selected timer and channel combination.

    \item \textbf{Enable Peripheral Clocks} \\
    Enable the RCC (Reset and Clock Control) for both the selected timer and the GPIO port that includes the chosen pin.

    \item \textbf{Configure the GPIO Pin} \\
    Set the pin to alternate function mode. Assign the appropriate alternate function number, and configure it as a push-pull output with an appropriate speed.

    \item \textbf{Configure the Timer} 
        \begin{itemize}[label=--]
        \item Set the timer to up-counting mode.
        \item Adjust the prescaler to set the base frequency.
        \item Set the auto-reload register (ARR) to define the PWM period.
        \item Set the compare register (CCR) to define the duty cycle.
    \end{itemize}

    \item \textbf{Enable PWM Output Mode} \\
    Configure the timer's output compare mode for the selected channel (PWM Mode 1 or 2) and enable preload so duty cycle changes take effect correctly.

    \item \textbf{Enable Timer Output (if required)} \\
    For advanced-control timers like TIM1, enable the main output by setting the MOE (Main Output Enable) bit.

    \item \textbf{Start the Timer} \\
    Enable the timer counter to start generating PWM signals on the configured pin.
\end{enumerate}

\end{document}
